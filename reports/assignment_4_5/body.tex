\textit{This brief report explains how we constructed the intermediate representation for the $\mu$C programming language and generated the concluding LLVM-assembler.}

\section{Intermediate Representation}

%TODO How is our IR represented? Structure etc.
%TODO Describe a simple IR-tree. Tikz picture?
%TODO How is the tree built?

\section{Code Generation}

%TODO How is the LLVM code generated out of the IR structure?
%TODO  - Walking down the IR-tree etc.

\subsection{LLVM Structure}

%TODO What is the general layout of LLVM code?
%TODO How does it compare to assembler such as MIPS?
%TODO What are the benefits of using LLVM instead of MIPS?

\subsection{Control Flow Statements}

%TODO How do we handle if-statements?
%TODO How do we handle while-statements?
%TODO What is the structure of these in IR and LLVM?
%TODO Basic blocks?

\subsection{Variable References}

%TODO Stack allocations
%TODO Global allocations
%TODO Arrays?
%TODO How are arrays passed in function calls?
%TODO How do we access regular allocations, store, load etc.
%TODO How do we access array-elements? store load etc
%TODO Temporaries and their counters

\section{Design Decisions}

%TODO Our solution to putint, putstring etc. Where we include a header in the LLVM code if needed.
%TODO How do we handle the creation of new basic blocks in LLVM after a return?

% - Semi-decisions, should we really describe these in this section?
%TODO How do we flatten arithmetic expressions? Are there any design decisions there?
%TODO How do we handle references to global variables inside functions? How do we determine if it is referencing a local or global?

\section{Usage}
%TODO How do you use the compiler?
%TODO What are the generated output files?
%TODO How do you execute the generated .ll files?
%TODO How do you generate a binary out of the .ll files?

\section{Conclusions}
